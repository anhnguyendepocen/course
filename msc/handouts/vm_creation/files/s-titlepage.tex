\maketitle
\setcounter{secnumdepth}{-2}
\begin{abstract}\noindent
This note outlines the process of creating a Virtual Machine (VM) using \textit{VirtualBox} and \textit{Vagrant} to teach basic software engineering skills. The VM is a software-based emulation of a computer running the \textit{Linux} based operating system \textit{Ubuntu Desktop 14.04 LTS}. Using a VM during class ensures the smooth installation of all required software tools and prepares students for the use of cloud computing environments. The choice of \textit{Ubuntu} allows students an easy transition to High Performance Computing (HPC) environments, which usually run on \textit{Linux} based operating systems as well.
\end{abstract}\vspace{0.3cm}

\tableofcontents
\thispagestyle{empty}
\vspace{0.5cm}
\renewcommand{\baselinestretch}{1.3}\normalsize	

\pagenumbering{arabic}
\setcounter{page}{1}\newpage
